\documentclass[11pt]{amsart}
\usepackage{graphicx} % Required for inserting images
\newcommand {\DS} {\displaystyle}
\newcommand{\diverg}{\nabla\cdot}
\newcommand{\curl}{\nabla \times}
\newcommand{\parder}[2]{\frac{\partial #1}{\partial #2}}
\newcommand{\der}[2]{\frac{d#1}{d#2}}
\usepackage{geometry} 
\usepackage{amsmath}
\geometry{a4paper}
\usepackage{amssymb}
\usepackage{epstopdf}
\usepackage{xcolor}
\usepackage{chemformula}
\title{Models Description}
\DeclareGraphicsRule{.tif}{png}{.png}{`convert #1 `dirname #1`/`basename #1 .tif`.png}

\title{Drift--Diffusion Equations with Chemical Reactions}


\begin{document}
\maketitle

\section{Research plan}
\begin{enumerate}
    \item rewrite test\_balcon\_et\_al.m replacing ode23s with ode15s. Write analytic Jacobian.
    \item extension to more than 2 species
    \item 1D problem. Constant reaction rates including drift
    \item 1D problem. Add the dependency of the reaction rates on the reduced electric field
    \item 1D problem. Add an energy equation for electron ($T_e$). $T_gas=const$. Reaction rates now depends on the electron temperature too
    \item 1D problem. Add the energy equation for the neutrals and heavy ions.
    \item ...
\end{enumerate}
\section{Reaction Systems}

Given the set of $M$ reactions involving $N$ species
\[
\sum_{k=1}^{N} r_{i,k}\ s_k\ =\ \sum_{k=1}^{N} p_{i,k}\ s_k \qquad  i=1\ldots M
\]
with forward reaction coefficient $c^f_i$ and backward reaction coefficient $c^b_i$,
the rate $R^{f}_i$ of the $i$-th forward reaction is 
$$
 R^{f}_i = c^{f}_i \prod_k s_k^{ r_{i,k} }
$$
the rate $R^{b}_i$ of the $i$-th forward reaction is 
$$
 R^{b}_i = c^{b}_i \prod_k s_k^{ p_{i,k} }
$$
the net rate of the i-th reaction is 
$$
R_i = R^{f}_i - R^{b}_i
$$
the system of odes representing the reactions is
$$
\dot{s}_k = \sum_{i=1}^M \left( - R^{f}_i r_{i,k} + R^{f}_i p_{i,k}
             - R^{b}_i p_{i,k} + R^{b}_i r_{i,k} \right) =
             - \sum_{i=1}^M  R_i\ (r_{i,k} - p_{i,k}) 
$$
The Jacobian of $\dot{s}_k$ is given by

$$
\dfrac{\partial \dot{s}_k}{\partial s_j} = -(r_{i,k} - p_{i,k}) \dfrac{\partial R_i}{\partial s_j}
$$

\section{Drift--Diffusion with Reactions}

The motion of the the $M$ species is governed by the continuity equations
$$
\dfrac{\partial s_{k}}{\partial t} + \mathrm{div} \left( \mathbf{F}_{k} \right) = \dot{s}_k  \qquad  i=1\ldots M
$$
where the flux $\mathbf{F}_{k}$ is composed of a Drift and a Diffusion part
$$
\mathbf{F}_{k} = - D_{k} \nabla s_{k} + \mu_{k}\ \mathbf{E}\ z_{k} s_{k}
$$
where $\mathbf{E}$ denotes the electric field, $z_{k}$ the valence number, $D_{k}$ and $\mu_{k}$ 
the diffusivity and mobility, respectively.
The electric field can be expressed as $\mathbf{E} = - \nabla \varphi$ and the electrostatic potential $\varphi$ 
satisfies
$$
-\mathrm{div}\ \left( \varepsilon \nabla \varphi \right) = q \sum_{k=1}^{N} z_{k} s_{k}
$$

\section{Models for the Reaction Coefficients}

$c^{b}_i$ and $c^{f}_i$ may depend on temperature, electric field and densities, \dots(add more details.. \textcolor{red}{\textbf{??}})

Arrhenius model prescribes the dependency of the reaction coefficients on the temperature
$$
c^{f}_i = A^{f}_{i} \exp\left( -\dfrac{\epsilon_a}{k_B T_{i}} \right)
$$
where $\epsilon_a$ is the Arrhenius activation energy.
A better experimental fitting can be obatined as suggest by Vincenti:
$$
c^{f}_i =A^{f}_{i} T^b \exp\left( -\dfrac{\epsilon_a}{k_B T_{i}} \right)
$$
The backward coefficient can be obtained as:
$$
c^{b}_i= c^{f}_i/C_c(T)  
$$
where $K_c(T)$ is obtained from the law of mass action:
$$
C_c(T)= \frac{\prod_{k=1}^N s_k^{p_{i,k}}}{\prod_{k=1}^N s_k^{r_{i,k}}}
$$

\section{Models for the Transport Coefficients}

$D_{k}$ and $\mu_{k}$ may depend on temperature, electric field and densities, \dots(add more details)
\textcolor{red}{\textbf{??}})
Usually they are related by the Stokes--Einstein relation 

$$
D_{k} = \dfrac{K_{b} T}{q} \mu_{k}
$$

\section{Energy/Temperature Equations}
$$
\parder{s_\epsilon}{t} + \diverg{\mathbf{{\Gamma_\epsilon}}} = H - P(\bar{\epsilon})
$$
where $s_\epsilon = \bar{\epsilon} s_e$ is the energy density, $\mathbf{\Gamma_e}$ is the electron mean energy flux, $H$ is the heating term, $P(\bar{\epsilon)}$ is the energy loss, and $\bar{\epsilon}$ is the electron mean energy.
The heating term, the energy loss and the energy flux are calculated as follows:
$$
H=-s_e e \mathbf{E} \cdot  \mathbf{\bar{v}}_e
$$ 
$$
P(\bar{\epsilon})=-s_e\bar{\nu}_e\bar{\epsilon}
$$
$$
\mathbf{\Gamma_\epsilon}= +s_\epsilon \mu_\epsilon z_k \mathbf{E} + D_\epsilon\nabla n_\epsilon
$$    
$\mathbf{\bar{v}}_e$ is the mean electron velocity, $\mu_\epsilon$ is the energy transfer frequency;
$\mu_\epsilon = 5/3\mu_e$ the energy mobility and $D_\epsilon = 5/3 D_e$ its diffusion coefficient.\\
Following the Balcon approach, the gas temperature is approximated to be constant and equal to the ambient one, therefore the energy equation is not needed under this hyphotesis.

\section{Definizione di file??}
\section{Description of the input file}
Given the generic reaction
$$
\nu_A A + \nu_B B \ch{<>} \nu_C C + \nu_D D    
$$
with $\nu_i$ stoichiometric coefficient for the $i-th$ species.\\
Il file Json riporta per ogni reazioni le informazioni secondo la seguente struttura:
\begin{itemize}
    \item reactants:\{"A": $\nu_A$, "B": $\nu_B$\}
    \item products:\{"C": $\nu_C$, "D": $\nu_D$\}
    \item model: Bolsig or Arrhenius)
    \item parameters for Arrhenius model A, b, $E_{a}$ respectively in an array of two elements for the reaction coefficient $c_{f}$(forward reaction) and  $c_{b}$ (backward reaction)
\end{itemize}

Let's make an example taking into account the Robertson system of reactions for the autocatalysis:
The system of reactions reads:

\begin{align}
    A \ch{->} B \\
    2B\ch{->} B + C\\
    B +C \ch{->} A + C
\end{align}
with rates:
$$
R_1^f=0.04 s_A
$$
$$
R_2^f=3*10^7 s_B^2
$$
$$
R_3^f=10^4 s_Bs_C
$$
the production rate is:
$$
\dot{s_A}=-R_1^f + R_3^f
$$
$$
\dot{s_B}=R_1^f - R_2^f-R_3^f
$$
$$
\dot{s_C}=R_2^f 
$$
The jacobian of this system is:
\begin{equation}
\begin{bmatrix} J
\end{bmatrix}=
\begin{bmatrix} -0.04&10^4s_C&10^4 s_B\\0.04&-(6*10^7+10^4s_C)s_B&-10^4s_B\\0&-6*10^7s_B&0
\end{bmatrix}
\notag
\end{equation}
This system of equation in the input file in $.json$ format reads:\\
\{\\
$"reactants" : \{ "A" : 1\}$\\
$"products" : \{ "B" : 1\},$\\
$"model": "Arrhenius",$\\
$"rate\_coeffs" : \{"A":0.04, "b":0,"E_a":0\}$\\
\}\\
\\
\{\\
$"reactants" : \{ "B" : 2 \},$\\
$"products" : \{ "B" : 1, "C" : 1 \}$,\\
$"model": "Arrhenius",$\\
$"rate\_coeffs" : \{"A":3e+7, "b":0,"E_a":0\}$\\
\}\\
\\
\{
$"reactants":\{ "B" : 1, "C" : 1 \}$\\
$"products" : \{ "A" : 1, "C" : 1 \},$\\
$"model": "Arrhenius",$\\
$"rate\_coeffs" : \{"A":1e+4, "b":0,"E_a":0\}$\\
\}\\

\section{Plasma chemistry in Argon}
The following reactions have been considered in the model(Balcon):
\begin{enumerate}
    \item Ionization by direct impact: $e + Ar \ch{<>} 2e + Ar^+$;
    \item Excitation: $e + Ar \ch{<>} e + Ar^*$;
    \item Ionization by direct impact on an excited atoms: $e + Ar^* \ch{<>} 2e + Ar^+$; 
    \item Ionization by collision of excited atoms: $2Ar^* \ch{<>} e + Ar^+ + Ar $;
    \item Associative ionization $Ar^+ + 2Ar \ch{<>} Ar + Ar_2^+$;
    \item Dissociative electron ion recombination $e + Ar_2^+ \ch{<>} Ar + Ar^*$;
    \item spontaneous de-excitation $Ar^* \ch{<>} Ar +h\nu$;
    \item electron-gas scattering $e + Ar \ch{<>} e + Ar$;
\end{enumerate}

The forward reaction rates are:
$$
R_1^f=c_1^f s_es_{Ar}
$$
$$
R_2^f=c_2^f s_es_{Ar}
$$
$$
R_3^f=c_3^f s_es_{Ar^*}
$$
$$
R_4^f=c_4^f s_{Ar^*}^2
$$
$$
R_5^f=c_5^f s_{Ar^+}s_{Ar}^2
$$
$$
R_6^f=c_6^f s_{e}s_{Ar_2^+}
$$
$$
R_7^f=c_7^f s_{Ar^*}
$$
$$
R_8^f=c_8^f s_es_{Ar}
$$
The backward reaction rates are:
$$
R_1^b=c_1^b s_e^2s_{Ar^+}
$$
$$
R_2^b=c_2^b s_es_{Ar^*}
$$
$$
R_3^b=c_3^b s_e^2s_{Ar^+}
$$
$$
R_4^b=c_4^b  s_es_{Ar}s_{Ar^+}
$$
$$
R_5^b=c_5^b s_{Ar}s_{Ar_2^+}
$$
$$
R_6^b=c_6^b s_{Ar}s_{Ar^*}
$$
$$
R_7^b=c_7^b s_{Ar}
$$
$$
R_8^b=c_8^b s_es_{Ar}
$$
with reaction coefficients specified according to the Arrhenius model of computed though BOLSIG+.

The system of ODEs is:
\begin{align*}
    \dot{s_e}= R_1 +R_3+ +R_4 -R_6\\
    \dot{s_{Ar}}= -R_1-R_2 +R_4- R_5+R_6+R_7\\
    \dot{s_{Ar^+}}=+R_1+R_3+R_4-R_5\\
    \dot{s_{Ar^*}}=+R_2-R_3-2R_4+R_6-R_7\\
    \dot{s_{Ar_2^+}}=+R_5-R_6
\end{align*}
in $.json$ format:\\
\{
$"reactants" : \{ "e" : 1, "Ar" : 1\},$\\
$"products" : \{ "e" : 2, "Ar+" : 1\},$\\
$"model" : \{"Bolsig": ["Ar", "Ar+"]\},$\\
$"rate_coeffs" : \{\}$\\
\}\\
\\
\{\\
	$"reactants" : \{ "e" : 1, "Ar" : 1\},$\\
	$"products" : \{ "e" : 1, "Ar*" : 1\},$\\
	$"model" : \{"Bolsig": ["Ar", "Ar*"]\},$\\
	$"rate_coeffs" : \{\}$\\
\}\\
\\
\{\\
	$"reactants" : \{ "e" : 1, "Ar*" : 1\},$\\
	$"products" : \{ "e" : 2, "Ar+" : 1\},$\\
	$"model" : \{"Bolsig": ["Ar*", "Ar+"]\},$\\
	$"rate_coeffs" : \{\}$\\
\}\\
\\
\{\\
	$"reactants" : \{ "Ar*" : 2\},$\\
	$"products" : \{ "e" : 1, "Ar+" : 1, "Ar" : 1\},$\\
	$"model" : "Arrhenius",$\\
$	"rate_coeffs" : \{"A" : 1.2e-9*300^0.5,"b": -0.5,"E_a": 0\}$\\
\}\\
\\
\{\\
	$"reactants" : \{ "Ar+" : 1, "Ar" : 2\},$\\
	$"products" : \{ "Ar2+" : 1, "Ar" : 1\},$\\
	$"model" : "Arrhenius",$\\
	$"rate_coeffs" : \{"A" : 2.5e-31*300^0.5,"b":-1.5,"E_a": 0\}$\\
\}\\
\\
\{\\
	$"reactants" : \{ "e" : 1, "Ar2+" : 1\},$\\
	$"products" : \{ "Ar*" : 1, "Ar" : 1\},$\\
	$"model" : "Arrhenius",	$\\
	$"rate_coeffs" : \{"A" : 7e-7*300^0.5,"b":-0.5,"E_a": 0\}$\\
\}\\
\\
\{\\
	$"reactants" : \{ "Ar*" : 1\},$\\
	$"products" : \{ "Ar" : 1, "h_nu" : 1\},$\\
	$"model" : "Arrhenius",$\\
	$"rate_coeffs" : \{"A" : 5e5,"b":0,"E_a": 0\}$\\
\}\\
\\
\{\\
	$"reactants" : \{ "e" : 1, "Ar" : 1\},$\\
	$"products" : \{ "e" : 1, "Ar" : 1\},$\\
	$"model" : \{"Bolsig": ["Ar", "Ar"]\},$\\
	$"rate_coeffs" : \{\}$\\
\}




\end{document}  